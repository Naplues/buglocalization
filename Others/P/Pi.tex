\documentclass{article}
\begin{document}
\title{Pi: A simple unsupervised approach to locating crash-inducing changes}
\maketitle
\paragraph{Abstract}
Background, Problem, Objective, Method, Result, Conclusion
Crash fault often occur on large projects may causes serious consequences. To find the root cause one crash, developers need to check lots of source codes to locate it, which is a time-consuming and tedious work confusing developers all the time. One feasible solution to the problem is to investigate software changes inducing a crash bug, called crash-inducing changes. By reviewing the these changes, developers can lock the primary cause where the crash was induced more easily.
In this paper, we propose Pi, a simple method to locate crash-inducing changes without supervised labels.
We evaluated Pi with six release versions of Netbeans project. The result show that the recall rate of Pi are close to ChangeLocator while the scores of MAP and MRR are higher than previous one.

\paragraph{Keywords}
Crash-inducing, Bug localization, Unsupervised approach

\section{Introduction}

\paragraph{}
Crash fault introduction.

\paragraph{}
Crash-inducing changes introduction.

\paragraph{}
ChangeLocator introduction.

\paragraph{}
The contributions of this paper.

\paragraph{}
The remainder of this paper is organized as follows. First, we briefly introduce the existing approach in Section 2. Section 3 presents our proposed Pi approach in details. We then design our experiment in Section 4 and show the evaluation results in Section 5. In Section 6 and Section 7, we discuss issues involved in our approach and the threats to validity, respectively. We 
illustrate the related work in Section 8 and conclude this paper in Section 9.

\section{Existing approach}

\section{Pi approach}

\section{Experimental design}

\section{Experimental result}

\section{Discussions}

\section{Threats to validity}

\section{Related work}

\section{Conclusions}

\section{Acknowledgments}

\section{References}


\end{document}